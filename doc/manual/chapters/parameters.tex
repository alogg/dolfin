\chapter{Parameters}
\label{sec:parameters}
\index{parameters}

\dolfin{} keeps a global database of parameters that control the
behavior of the various components of \dolfin{}. Parameters are
controlled using a uniform type-independent interface that allows
retrieving the values of existing parameters, modifying existing
parameters and adding new parameters to the database.

%------------------------------------------------------------------------------
\section{Retrieving the value of a parameter}
\index{\texttt{dolfin\_get()}}

To retrieve the value of a parameter, use the function \texttt{dolfin\_get()}
available in the \texttt{dolfin} namespace:
\begin{code}
  Parameter dolfin_get(const char* key);
\end{code}
This function accepts as argument a string \texttt{key} and returns
the value of the parameter matching the given key. An error message is
printed through the log system if there is no parameter with the given
key in the database.

The value of the parameter is automatically cast to the correct type
when assigning the value of \texttt{dolfin\_get()} to a variable, as
illustrated by the following examples:
\begin{code}
  real TOL = dolfin_get(``tolerance'');
  int num_samples = dolfin_get(``number of samples'');
  bool solve_dual = dolfin_get(``solve dual problem'');
  std::string filename = dolfin_get(``file name'');
\end{code}

Note that there is a cost associated with accessing the value of a
parameter, so if the value of a parameter is to be used multiple
times, then it should be retrieved once and stored in a local variable
as illustrated by the following example:
\begin{code}
  int num_samples = dolfin_get(``number of samples'');
  for (int i = 0; i < num_samples; i++)
  \{
    ...
  \}
\end{code}

%------------------------------------------------------------------------------
\section{Modifying the value of a parameter}
\index{\texttt{dolfin\_set()}}

To modify the value of a parameter, use the function \texttt{dolfin\_set()}
available in the \texttt{dolfin} namespace:
\begin{code}
  void dolfin_set(const char* key, ...);
\end{code}
This function accepts as arguments a string \texttt{key} together with
the corresponding value. The value type should match the type of
parameter that is being modified. An error message is
printed through the log system if there is no parameter with the given
key in the database.

The following examples illustrate
the use of \texttt{dolfin\_set()}:
\begin{code}
  dolfin_set(``tolerance'', 0.01);
  dolfin_set(``number of samples'', 10);
  dolfin_set(``solve dual problem'', true);
  dolfin_set(``file name'', ``solution.xml'');
\end{code}

Note that changing the values of parameters using
\texttt{dolfin\_set()} does not change the values of already retrieved
parameters; it only changes the values of parameters in the
database. Thus, the value of a parameter must be changed before using
a component that is controlled by the parameter in question.

%------------------------------------------------------------------------------
\section{Adding a new parameter}
\index{\texttt{dolfin\_parameter()}}

To add a parameter to the database, use the function
\texttt{dolfin\_parameter()} available in the \texttt{dolfin}
namespace:
\begin{code}
  void dolfin_parameter(Parameter::Type type, 
                        const char* key, ...);
\end{code}
This function accepts three arguments: the type of the new parameter,
a unique key identifying the new parameter and the value of the new
parameter.

Possible values for \texttt{type} are
\begin{itemize}
\item
  \texttt{Parameter::REAL}, corresponding to \texttt{real};
\item
  \texttt{Parameter::INT}, corresponding to \texttt{int};
\item
  \texttt{Parameter::BOOL}, corresponding to \texttt{bool};
\item
  \texttt{Parameter::STRING}, corresponding to \texttt{std::string}.
\end{itemize}

The following examples illustrate the use of
\texttt{dolfin\_parameter()}:
\footnotesize
\begin{code}
  dolfin_parameter(Parameter::REAL, ``tolerance'', 0.01);
  dolfin_parameter(Parameter::INT, ``number of samples'', 10);
  dolfin_parameter(Parameter::BOOL, ``solve dual problem'', true);
  dolfin_parameter(Parameter::STRING, ``file name'', ``solution.xml'');
\end{code}
\normalsize

%------------------------------------------------------------------------------
\section{Saving parameters to file}
\index{\texttt{dolfin\_save()}}
\index{XML}

To save the current database of parameters to a file in \dolfin{} XML
format, use the function \texttt{dolfin\_save()} available in the
\texttt{dolfin} namespace:
\begin{code}
  void dolfin_save(const char* filename);
\end{code}
When running a simulation in \dolfin{}, saving the parameter database
to a file is an easy way to document the set of parameters used in the
simulation.

%------------------------------------------------------------------------------
\section{Loading parameters from file}
\index{\texttt{dolfin\_load()}}
\index{XML}

To load a set of parameters from a file into the parameter database,
use the function \texttt{dolfin\_load()} available in the
\texttt{dolfin} namespace:
\begin{code}
  void dolfin_load(const char* filename);
\end{code}
This function accepts as argument the name of a file containing a list
of a parameters in \dolfin{} XML format, as illustrated below:
\footnotesize
\begin{code}
<?xml version=''1.0'' encoding=''UTF-8''?> 

<dolfin xmlns:dolfin=''http://www.fenics.org/dolfin/''> 
  <parameters>
    <parameter name=''tolerance'' type=''real'' value=''0.01''/>
    <parameter name=''number of samples'' type=''int'' value=''10''/>
    <parameter name=''solve dual problem'' type=''bool'' value=''false''/>
    <parameter name=''file name'' type=''string'' value=''solution.xml''/>
  </parameters>
</dolfin>
\end{code}
\normalsize
