\chapter{Coding style}
\label{sec:codingstyle}
\index{coding style}

To streamline the \dolfin{} source code and ease the job for
maintainers that need to read and edit large amounts of code,
developers should try to follow the below coding style when submitting
patches to \dolfin{}.

The guideline below is for C++ but may in some cases be extrapolated
to Python.

\section{Naming conventions}

\subsection{Class names}

Use camel caps for class names:
\begin{code}
class FooBar
{
  ...
};
\end{code}

\subsection{Function names}

Use lower-case for function names and underscore to separate words:
\begin{code}
void foo();
void bar();
void foo_bar(...);
\end{code}

Functions returning a value should be given the name of that value,
for example:
\begin{code}
class Array:
{
public:

  /// Return size of array (number of entries)
  uint size() const;

};
\end{code}

In the above example, the function should be named \texttt{size} rather
than \texttt{get\_size}. On the other hand, a function not returning a
value but rather taking a variable (by reference) and assigning a value
to it, should use the \texttt{get\_foo} naming scheme, for example:
\begin{code}
class Parameters:
{
public:

  /// Retrieve all parameter keys
  void get_parameter_keys(std::vector<std::string>& parameter_keys) const;

};
\end{code}

\subsection{Variable names}

Use lower-case for variable names and underscore to separate words:
\begin{code}
Foo foo;
Bar bar;
FooBar foo_bar;
\end{code}

\subsection{Enum variables and constants}

Enum variables should be lower-case with underscore to separate words:
\begin{code}
enum Type {foo, bar, foo_bar};
\end{code}

We try to avoid using \texttt{\#define} to define constants, but when
necessary constants should be capitalized:
\begin{code}
#define FOO 3.14159265358979
\end{code}

\subsection{File names}

Use camel caps for file names if they contain the
declaration/definition of a class. Header files should have the
suffix~\texttt{.h} and implementation files should have the
suffix~\texttt{.cpp}:
\begin{code}
FooBar.h
FooBar.cpp
\end{code}

Use lower-case for file names that contain utilities/functions (not
classes).

\section{Miscellaneous}

\subsection{Comments}

Comment your code, and do it often. Capitalize the first letter and
don't use punctuation (unless the comment runs over several
sentences). Here's a good example from \texttt{TopologyComputation.cpp}:
\begin{code}
// Check if connectivity has already been computed
if (connectivity.size() > 0)
  return;

// Invalidate ordering
mesh._ordered = false;

// Compute entities if they don't exist
if (topology.size(d0) == 0)
  computeEntities(mesh, d0);
if (topology.size(d1) == 0)
  computeEntities(mesh, d1);

// Check if connectivity still needs to be computed
if (connectivity.size() > 0)
  return;

...
\end{code}

\subsection{Integers and reals}

Use \texttt{dolfin::uint} instead of \texttt{int} (unless you really
want to use negative integers which is rare) and \texttt{dolfin::real}
instead of \texttt{double}:
\begin{code}
uint i = 0;
double x = 0.0;
\end{code}
These are typedefs for the standard C++ types \texttt{unsigned int}
and \texttt{double} (defined in \texttt{dolfin/common/types.h}).

\subsection{Placement of brackets}

Curly brackets following a control statement should appear in the next
line and not be indented:
\begin{code}
for (uint i = 0; i < 10; i++)
{
  ...
}
\end{code}

\subsection{Indentation}

Indentation should be two spaces and it should be spaces, \emph{not}
tab(s).

\subsection{Header file layout}

Header files should follow the below template:
\vspace{-0.5cm}
\begin{code}
// Copyright (C) 2008 Foo Bar.
// Licensed under the GNU LGPL Version 2.1.
//
// Modified by Bar Foo, 2008.
//
// First added:  2008-01-01
// Last changed: 2008-02-01

#ifndef __FOO_H
#define __FOO_H

namespace dolfin
{

  class Bar; // Forward declarations here

  /// Documentation of class

  class Foo
  {
  public:

    ...

  private:

    ...

  };

}

#endif
\end{code}

\subsection{Implementation file layout}

Implementation files should follow the below template:
\begin{code}
// Copyright (C) 2008 Foo Bar.
// Licensed under the GNU LGPL Version 2.1.
//
// Modified by Bar Foo, 2008.
//
// First added:  2008-01-01
// Last changed: 2008-02-01

#include <dolfin/Foo.h>

using namespace dolfin;

//-----------------------------------------------------------
Foo::Foo() : // variable initialization here
{
  ...
}
//-----------------------------------------------------------
Foo::~Foo()
{
  // Do nothing
}
//-----------------------------------------------------------
\end{code}

The horizontal lines above should be exactly~79 characters
wide but have been shortened here to fit the page.

\subsection{Including header files}

Don't use \texttt{\#include <dolfin.h>} or \texttt{\#include
  <dolfin/dolfin\_foo.h>} inside the DOLFIN kernel. Only include the
portions of DOLFIN you are actually using.

\subsection{Forward declarations}

Actually, try to include as little as possible and use forward
declarations whenever possible (in header files). Put the
\texttt{\#include} in the implementation file.

\subsection{Explicit constructors}

Make all constructors (except copy constructors) explicit if there is no particular
reason not to do so:
\begin{code}
class Foo
{
  explicit Foo(uint i);
};
\end{code}

\subsection{Virtual functions}

Always declare inherited virtual functions as virtual in the subclasses. This makes it
easier to spot which functions are virtual.

\begin{code}
class Foo
{
  virtual void foo();
  virtual void bar() = 0;
};

class Bar
{
  virtual void foo();
  virtual void bar();
};
\end{code}

\section{Use of libraries}

\subsection{Prefer C++ strings and streams to old C-style \texttt{char*}}

Use std::string instead of \texttt{const char*} and use std::istream and
std::ostream instead of \texttt{FILE}. Avoid \texttt{printf},
\texttt{sprintf} and the like.

There are exceptions to this rule where we need to use old C-style
function calls. One such exception is handling of command-line
arguments (\texttt{char* argv[]}).
