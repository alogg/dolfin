\chapter{Functions}
\index{functions}
\index{Function}

This chapter discusses the representation of functions in \dolfin{}. There are two type of functions; functions which belong to the finite element space and user-defined functions.



\section{Functions in the finite element space}
%
A function in in the finite element space is defined in terms of a finite element, a mesh and a vector containing the the value of the function at a points (nodal points). For example, a scalar $u$ two dimensions is given by:
\begin{equation}
  u(x,y) = \sum_{i=1}^{n} N_{i}(x,y) u_{i},
\end{equation}
where $N_{i}$ is the finite element shape function (basis function) associated with 
node~$i$, $u_{i}$ is the value of the function at node~$i$ and $n$ is the number of nodes.


\section{User-defined functions}
\index{user-defined functions}
%
A user defined function is a function in terms of the spatial coordinates and possibly time. They are typically used for defining source terms and initial conditions. For example, a source could be given by
\begin{equation}
  f = f(x,y) = y \sin(x / \pi).
\end{equation}

\devnote{This chapter is currently being written\ldots}

% FIXME: Discuss the Function class and the different representations.
