\chapter{Installation}
\label{app:installation}
\index{installation}

The source code of \dolfin{} is portable and should compile on any
Unix system, although it is developed mainly under GNU/Linux (in 
particular Debian GNU/Linux). \dolfin{} can be compiled under Windows
through Cygwin~\cite{www:Cygwin}. Questions, bug reports and patches
concerning the installation should be directed to the \dolfin{} mailing 
list at the address
\begin{code}
  dolfin-dev@fenics.org
\end{code}

\dolfin{} must currently be compiled directly from source, but an effort
is underway to provide precompiled Debian packages of \dolfin{} and
other \fenics{} components.

%------------------------------------------------------------------------------
\section{Installing from source}

\subsection{Dependencies and requirements}
\index{dependencies}

\dolfin{} depends on a number of libraries that need to be installed on your
system. These libraries include Libxml2 and PETSc. In addition to these 
libraries, you need to install \fiat{} and \ffc{} if you want to define your 
own variational forms.

\subsubsection{Installing Libxml2}
\index{Libxml2}

Libxml2 is a library used by \dolfin{} to parse XML data files. Libxml2 can be
obtained from
\begin{code}
  http://xmlsoft.org/
\end{code}
Packages are available for most Linux distributions. For Debian users, the 
package to install is \texttt{libxml2-dev}.

\subsubsection{Installing PETSc}
\index{PETSc}

PETSc is a library for the solution of linear and nonlinear systems, functioning
as the backend for the \dolfin{} linear algebra classes. \dolfin{} depends on 
PETSc version 2.3.1, which can be obtained from
\begin{code}
  http://www-unix.mcs.anl.gov/petsc/petsc-2/
\end{code}

Follow the installation instructions on the PETSc web page. Normally,
you should only have to perform the following simple steps in the PETSc
source directory:
\begin{code}
  # export PETSC_DIR=`pwd`
  # ./config/configure.py --with-clanguage=cxx --with-shared=1
  # make all
\end{code}

Add \texttt{--download-hypre=yes} to configure.py if you want to
install Hypre which provides a collection of preconditioners,
including algebraic multigrid (AMG), and 
\texttt{--download-umfpack=yes} to configure.py if you want to
install UMFPACK which provided as fast direct linear solver.
Both packages are highly recommended.

DOLFIN assumes that \texttt{PETSC\_DIR} is \texttt{/usr/local/lib/petsc/} but
this can be controlled using the flag \texttt{--with-petsc-dir=<path>} when 
configuring DOLFIN (see below).

\subsubsection{Installing FFC}
\index{FFC}

\dolfin{} uses the FEniCS Form Compiler \ffc{} to process variational
forms. \ffc{} can be obtained from
\begin{code}
  http://www.fenics.org/
\end{code}

Follow the installation instructions given in the \ffc{}
manual. \ffc{} follows the standard for Python packages, which means
that normally you should only have to perform the following simple step
in the \ffc{} source directory:
\begin{code}
  # python setup.py install
\end{code}

Note that \ffc{} depends on \fiat{} \index{FIAT}, which in turn depends on
the Python packages Numeric (Debian package \texttt{python-numeric}) and
LinearAlgebra (Debian package \texttt{python-numeric-ext}). Refer to
the \ffc{} manual for further details.

% Input section shared with FFC manual
% This chapter is common to the DOLFIN and FFC manuals.

\subsection{Downloading the source code}
\index{downloading}
\index{source code}

The latest release of \package{} can be obtained as a \texttt{tar.gz}
archive in the download section at
\begin{code}
 http://www.fenics.org/
\end{code}

Download the latest release of \package{}, for example \texttt{\packagett{}-x.y.z.tar.gz},
and unpack using the command
\begin{macrocode}
# tar zxfv \packagett{}-x.y.z.tar.gz
\end{macrocode}

This creates a directory \texttt{\packagett{}-x.y.z} containing the
\package{} source code.

If you want the very latest version of \package{}, it can be accessed
directly from the development repository through \texttt{hg}
(Mercurial):
\begin{macrocode}
# hg clone http://www.fenics.org/hg/\packagett{}
\end{macrocode}
This version may contain features not yet present in the latest
release, but may also be less stable and even not work at all.


\subsection{Compiling the source code}
\index{compiling}

\dolfin{} is built using the standard GNU Autotools (Automake,
Autoconf) and libtool, which means that the installation procedure is simple:
\begin{code}
  # ./configure
  # make
\end{code}
followed by an optional
\begin{code}
  # make install
\end{code}
to install \dolfin{} on your system.

The configure script will check for a number of libraries and try
to figure out how compile \dolfin{} against these libraries. The
configure script accepts a collection of optional arguments that can be
used to control the compilation process. A few of these are listed
below. Use the command
\begin{code}
  # ./configure --help
\end{code}
for a complete list of arguments.

\begin{itemize}
\item
  Use the option \texttt{--prefix=<path>} to specify an alternative
  directory for installation of \dolfin{}. The default directory is
  \texttt{/usr/local/}, which means that header files will be
  installed under \texttt{/usr/local/include/} and libraries will be
  installed under \texttt{/usr/local/lib/}. This option can be useful
  if you don't have root access but want to install \dolfin{} locally
  on a user account as follows:
  \begin{code}
    # mkdir ~/local
    # ./configure --prefix=~/local
    # make
    # make install
  \end{code}
\item
  Use the option \texttt{--enable-debug} to compile \dolfin{} with
  debugging symbols and assertions.
\item
  Use the option \texttt{--enable-optimization} to compile an
  optimized version of \dolfin{} without debugging symbols
  and assertions.
\item
  Use the option \texttt{--disable-curses} to compile \dolfin{}
  without the curses interface (a text-mode graphical user interface).
\item
  Use the option \texttt{--disable-mpi} to compile \dolfin{} without
  support for MPI (Message Passing Interface), assuming PETSc has been
  compiled without support for MPI.
\item
  Use the option \texttt{--with-petsc-dir=<path>} to specify the
  location of the PETSc directory. By default, \dolfin{} assumes that
  PETSc has been installed in \texttt{/usr/local/lib/petsc/}.
\end{itemize}

\subsection{Compiling the demo programs}
\index{demo programs}

After compiling the \dolfin{} library according to the instructions
above, you may want to try one of the demo programs in the
subdirectory \texttt{src/demo/} of the \dolfin{} source tree.
Just enter the directory containing the demo program you want to
compile and type \texttt{make}. You may also compile all demo programs
at once using the command
\begin{code}
  # make demo
\end{code}

\subsection{Compiling a program against \dolfin{}}
\index{compiling}

Whether you are writing your own Makefiles or using an automated build
system such as GNU Autotools or BuildSystem, it is straightforward to
compile a program against \dolfin{}. The necessary include and library
paths can be obtained through the script \texttt{dolfin-config} which
is automatically generated during the compilation of \dolfin{} and
installed in the \texttt{bin} subdirectory of the \texttt{<path>}
specified with \texttt{--prefix}. Assuming this directory is in your
executable path (environment variable \texttt{PATH}), the include
path for building \dolfin{} can be obtained from the command
\begin{code}
  dolfin-config --cflags
\end{code}
and the path to \dolfin{} libraries can be obtained from the command
\begin{code}
  dolfin-config --libs
\end{code}
If \texttt{dolfin-config} is not in your executable path, you need to
provide the full path to \texttt{dolfin-config}.

Examples of how to write a proper \texttt{Makefile} are provided with
each of the example programs in the subdirectory \texttt{src/demo/} in
the \dolfin{} source tree.

%------------------------------------------------------------------------------
\section{Debian package}
\index{Debian package}

In preparation.

%------------------------------------------------------------------------------
\section{Installing from source under Windows}
\label{app:cygwin}
\index{Cygwin}

\dolfin{} can be used under Windows using Cygwin, which provides a Linux-like
environment. The installation process is the same as under GNU/Linux. 
To use \dolfin{} under Cygwin, the Cygwin development tools must be installed. 
Instructions for installing PETSc under Cygwin can be found on the  PETSc web 
page. Installation of \ffc{} and \fiat{} is the same as under GNU/Linux. The 
Python package Numeric is not available as a Cygwin package and must be 
installed manually. To compile \dolfin{}, the Cygwin package  
\texttt{libxml2-devel} must be installed. The compilation procedure is then the
same as under GNU/Linux. If MPI has not been installed:
\begin{code}
  # ./configure --disable-mpi
  # make
\end{code}
followed by an optional
\begin{code}
  # make install
\end{code}
will compile \dolfin{} on your system.
