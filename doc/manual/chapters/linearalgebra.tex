\chapter{Linear algebra}

\dolfin{} uses PETSc for the linear algebra. For convenience \dolfin{} provide 
wrappers for some of the most common linear algebra functionality.   

\section{The Matrix class}

The matrix class represents a matrix of dimension m x n. It is a
simple wrapper for a PETSc matrix \texttt{Mat}. The interface is
intentionally simple. For advanced usage, access the PETSc Mat
pointer using the function mat() and use the standard PETSc
interface.

\section{The VirtualMatrix class}

This class represents a matrix-free matrix of dimension m x m.
It is a simple wrapper for a PETSc shell matrix. The interface
is intentionally simple. For advanced usage, access the PETSc
Mat pointer using the function mat() and use the standard PETSc
interface.

The class VirtualMatrix enables the use of Krylov subspace
methods for linear systems Ax = b, without having to explicitly
store the matrix A. All that is needed is that the user-defined
VirtualMatrix implements multiplication with vectors. Note that
the multiplication operator needs to be defined in terms of
PETSc data structures (\texttt{Vec}), since it will be called from PETSc.


\section{The Vector class}

The vector class represents a vector of dimension n. It is a
simple wrapper for a PETSc vector \texttt{Vec}. The interface is
intentionally simple. For advanced usage, access the PETSc Vec
pointer using the function vec() and use the standard PETSc
interface.

\section{The LinearSolver class}

This class defines the interface of all linear solvers for
systems of the form Ax = b.

\section{The GMRES class}

This class implements the GMRES method for linear systems
of the form Ax = b. It is a wrapper for the GMRES solver
of PETSc.

\section{The LU class}

This class implements the direct solution (LU factorization) for
linear systems of the form Ax = b. It is a wrapper for the LU
solver of PETSc.

\section{The EigenvalueSolver class}

This class computes eigenvalues of a matrix. It is 
a wrapper for the eigenvalue solver of PETSc.

\section{The Preconditioner class}

This class specifies the interface for user-defined Krylov
method preconditioners. A user wishing to implement her own
preconditioner needs only supply a function that approximately
solves the linear system given a right-hand side.

\section{The PETScManager class}

This class is responsible for initializing and (automatically)
finalizing PETSc. To initialize PETSc, call PETScManager::init()
once (additional calls will be ignored). Finalization will be
handled automatically.

\section{The PETSc system}

PETSc is a suite of data structures and routines for the scalable 
(parallel) solution of scientific applications modeled by partial 
differential equations.  It employs the MPI standard for all message-passing communication.  

\section{The Hypre system}

As a complement to PETSc, \dolfin{} also uses Hypre, which is a 
library for solving large, sparse linear systems of equations on massively parallel computers. 

To use a preconditioner from Hypre with your PETSc solver in \dolfin{}, write 
\begin{code}
PCSetType(PC pc, PCHYPRE );
PCHYPRESetType(PC pc,"boomeramg");
\end{code}

In particular, the above preconditioner \texttt{boomeramg} is an algebraic multigrid 
preconditioner, which may be very useful for some problems. 

\fixme{Write about the wrappers, PETSc, using mat() and vec() to do more advanced operations with PETSc etc.}
