\chapter{The log system}

\dolfin{} provides provides a simple interface for uniform handling of
log messages, including warnings and errors. All messages are
collected to a single stream which allows the destination and
formatting of the output from a entire program, including the
\dolfin{} library, to be controlled by the user.

%------------------------------------------------------------------------------
\section{Generating log messages}

Log messages can be generated using the function
\texttt{dolfin\_info()} available in the \texttt{dolfin} namespace:
\begin{code}
  void dolfin_info(const char *msg, ...);
\end{code}
which works similarly to the standard C library function \texttt{printf}.
The following examples illustrate the usage of
\texttt{dolfin\_info()}:
\begin{code}
  dolfin_info(``Solving linear system.'');
  dolfin_info(``Size of vector: \%d.'', x.size());
  dolfin_info(``R = \%.3e (TOL = \%.3e)'', R, TOL);
\end{code}

As an alternative to \texttt{dolfin\_info()}, \dolfin{} provides a C++
style interface to generating log messages. Thus, the above examples
can also be implemented as follows:
\footnotesize
\begin{code}
  cout << ``Solving linear system.'' << endl;
  cout << ``Size of vector: `` << x.size() << ``.'' << endl;
  cout << ``R = `` << R << `` (TOL = `` << TOL << ``)'' << endl;
\end{code}
\normalsize
using \texttt{dolfin::cout} and \texttt{dolfin::endl} from the \texttt{dolfin}
namespace, corresponding to the standard standard \texttt{std::cout}
and \texttt{std::endl} in namespace \texttt{std}. If log messages are
directed to standard output (see below), then \texttt{dolfin::cout}
and \texttt{std::cout} may be mixed freely.

%------------------------------------------------------------------------------
\section{Warnings and errors}

Warnings and error messages can be generated using the macros
\begin{code}
  dolfin_warning(message);
  dolfin_error(message);
\end{code}

In addition to displaying the given string message, the macro
\texttt{dolfin\_error()} also displays information about the location
of the code that generated the error (file, function name and line
number). Once an error is encountered, the program is stopped.

Note that in order to pass formatting strings and additional arguments
to warnings or errors, the variations \texttt{dolfin\_error1()},
\texttt{dolfin\_error2()} and so on must be used, as illustrated by
the following examples:
\footnotesize
\begin{code}
  dolfin_error(``GMRES solver did not converge.'');
  dolfin_error1(``Unable to find face opposite to node %d.'', n);
  dolfin_error2(``Unable to find edge between nodes %d and %d.'', n0, n1);
\end{code}
\normalsize

%------------------------------------------------------------------------------
\section{Debug messages and assertions}

The macro \texttt{dolfin\_debug()} works similarly to
\texttt{dolfin\_info()}:
\begin{code}
  dolfin_debug(message);
\end{code}
but in addition to displaying a message, information is printed about
the location of the code that generated the debug message (file,
function name and line number).

Note that in order to pass formatting strings and additional arguments
with debug messages, the variations \texttt{dolfin\_debug1()},
\texttt{dolfin\_debug2()}, depending on the number of arguments.

Assertions can often be a helpful programming tool. Use assertions
whenever you assume something about about a variable in your code,
such as checking that given input to a function is valid. \dolfin{}
provides the macro \texttt{dolfin\_assert()} for creating assertions:
\begin{code}
  dolfin\_assert(check);
\end{code}
This function accepts a boolean expression and if the expression
evaluates to false, an error message is displayed, including the
file, function name and line number of the assertion, and a
segmentation fault is raised (to enable easy attachment to a
debugger). The following examples illustrate the use of
\texttt{dolfin\_assert()}:
\begin{code}
  dolfin_assert(i >= 0);
  dolfin_assert(i < n);
  dolfin_assert(cell.type() == Cell::triangle);
  dolfin_assert(cell.type() == Cell::tetrahedron);
\end{code}
Note that assertions are only active if \dolfin{} when compiling
\dolfin{} and your program with \texttt{DEBUG} defined (configure
option \texttt{--enable-debug} or compiler flag \texttt{-DDEBUG}).
Otherwise, the macro \texttt{dolfin\_assert()} expands to nothing,
meaning that liberal use of assertions does not mean a performance
penalty, since assertions only are present during development and
debugging.

%------------------------------------------------------------------------------
\section{Task notification}

The two function \texttt{dolfin\_begin()} and \texttt{dolfin\_end()}
available in the \texttt{dolfin} name space can be used to notify the
\dolfin{} log system about the beginning and end
The \dolfin{} log system indents log messages hierarchically 
To notify the \dolfin{} log system about the beginning of 
\dolfin{} provides two functions 
%// Task notification
%namespace dolfin { void dolfin_begin(); }
%namespace dolfin { void dolfin_begin(const char* msg, ...); }
%namespace dolfin { void dolfin_end(); }
%namespace dolfin { void dolfin_end(const char* msg, ...); }

%------------------------------------------------------------------------------
\section{Progress bars}
\index{progress bar}

To notify progress by a progress session, use the class
Progress.

Examples of usage:

\begin{code}
  Progress p("Assembling", grid.noCells());
  
  for (CellIterator c(grid); !c.end(); ++c) {
    ...
    p++;
  }
\end{code}

Progress also supports the following usage:

\begin{code}
  p = i;    // Specify step number
  p = 0.5;  // Specify percentage
  p.update(t/T, "Time is t = %f", t);
\end{code}

%------------------------------------------------------------------------------
\section{Controlling the destination of output}

% curses, terminal, silent
%
% // Update (force refresh of curses interface)
% namespace dolfin { void dolfin_update(); }
%
%
%// Specify output type (``plain text'', ``curses'', or ``silent'')
%void dolfin_output(const char* type);
%
%// Switch logging on or off
%void dolfin_log(bool state);



