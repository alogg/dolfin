\addcontentsline{toc}{chapter}{About this manual}
\chapter*{About this manual}

This manual is currently being written. As a consequence, some
sections may be incomplete or inaccurate. In particular, only the C++
interface (not the Python interface) of \dolfin{} is documented, and
only to a certain extent. Care has been taken that the quickstart
chapter is accurate, but other than that, inconsistencies and
inaccuracies can be expected.

We apologize for any inconvenience, but take comfort in the fact that
(i) with the release of \dolfin{} 0.7.0, the interface is starting to
mature and will undergo less dramatic changes in the future (which
will actually make it possible to write documentation) and (ii) most
of the code is pretty well documented through the demos. If you have
some writing skills and are willing to contribute, please consider
writing a section or two and submit to the mailing list!

%------------------------------------------------------------------------------
\section*{Intended audience}

This manual is written both for the beginning and the advanced user.
There is also some useful information for developers. More advanced topics
are treated at the end of the manual or in the appendix.

%------------------------------------------------------------------------------
\section*{Typographic conventions}
\index{typographic conventions}

\begin{itemize}
\item
  Code is written in monospace (typewriter) \texttt{like this}.
\item
  Commands that should be entered in a Unix shell
  are displayed as follows:
  \begin{code}
# ./configure
# make
  \end{code}
  Commands are written in the dialect of the \texttt{bash} shell. For
  other shells, such as \texttt{tcsh}, appropriate translations may be
  needed.
\end{itemize}

%------------------------------------------------------------------------------
\section*{Enumeration and list indices}
\index{enumeration}
\index{indices}

Throughout this manual, elements $x_i$ of sets $\{x_i\}$ of size $n$
are enumerated from $i = 0$ to $i = n-1$. Derivatives in $\R^n$ are
enumerated similarly:
$\frac{\partial}{\partial x_0}, \frac{\partial}{\partial x_1},
 \ldots, \frac{\partial}{\partial x_{n-1}}$.

%------------------------------------------------------------------------------
\section*{Contact}
\index{contact}

Comments, corrections and contributions to this manual are most welcome
and should be sent to
\begin{macrocode}
\packagett{}-dev@fenics.org    
\end{macrocode}
